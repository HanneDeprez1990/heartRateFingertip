%%%%%%%%%%%%%%%%%%%%%%%%%%%%%%%%%%%%%%%%%%%%%%%%%%%%%%%%%%%%%%%%%%%%%%%
%%%%%%%%%%%%%%%%%%%%%%%%%%%%%%%%%%%%%%%%%%%%%%%%%%%%%%%%%%%%%%%%%%%%%%%
\deelmetoef{Module 4}{Automatische piekdetectie}{Module 4. Automatische piekdetectie}{Oplossingen module 4}{Oplossingen module 4}
\label{sec:piekdetectie}
%%%%%%%%%%%%%%%%%%%%%%%%%%%%%%%%%%%%%%%%%%%%%%%%%%%%%%%%%%%%%%%%%%%%%%%
%%%%%%%%%%%%%%%%%%%%%%%%%%%%%%%%%%%%%%%%%%%%%%%%%%%%%%%%%%%%%%%%%%%%%%%

\begin{samenvatting}
	In de vorige modules leerde je werken met App Inventor 2 en leerde je de basis van digitale afbeeldingen en video. In de slotopdracht van module 3 maakte je een grafiek waaruit je je hartslag kon afleiden door zelf pieken te tellen. In deze module gaan we na hoe de smartphone \emph{zelf} die pieken kan detecteren, zodat wij van dat rekenwerk gespaard blijven. Daarvoor staan we stil bij wat een piek precies is, en hoe we de smartphone kunnen leren pieken te detecteren. 
\end{samenvatting}
%

%%%%%%%%%%%%%%%%%%%%%%%%%%%%%%%%%%%%%%%%%%%%%%%%%%%%%%%%%%%%%%%%%%%%%%%
\section{Wat is een piek in een signaal?}
\label{sec:Mod4_Sec1}
%%%%%%%%%%%%%%%%%%%%%%%%%%%%%%%%%%%%%%%%%%%%%%%%%%%%%%%%%%%%%%%%%%%%%%%
%
I.p.v. zelf pieken te tellen op een grafiek willen we dat de app dit automatisch doet. Hiervoor zullen we de code moeten schrijven om pieken te detecteren. Als mens kunnen wij eenvoudig en redelijk intu\"itief pieken herkennen. De smartphone kan dat niet uit zichzelf; we zullen moeten \textquotedblleft uitleggen\textquotedblright \ wanneer hij een extra piek moet tellen. Om dit goed te kunnen uitleggen, staan we eerst even stil bij wat een piek nu eigenlijk is.

\begin{steroef}
	Omschrijf in eigen woorden wat een piek is.
\end{steroef}
\oplos{Wellicht komen hier heel diverse antwoorden, afhankelijk van de achtergrond van de leerlingen. Leerlingen uit de 3e graad zullen wellicht over maxima spreken, terwijl leerlingen uit de 2e graad misschien meer woordelijke/minder wiskundige omschrijvingen zullen geven.}

Een woordelijke omschrijving van een piek hebben we gevonden, maar je smartphone redeneert wiskundiger. 

\begin{oef}
	Stel een wiskundige uitdrukking op die je in de code kan gebruiken om pieken te detecteren. 
	
	Hou er rekening mee dat je smartphone enkel een lijst met roodwaarden als input heeft. Die lijst kunnen we voorstellen als een verzameling $\{ u_1, u_2, u_3, \ldots, u_n \}$ met $n$ elementen, waarbij $u_i$ de $i$-de berekende gemiddelde roodwaarde voorstelt.
	
	Als je dit niet meteen ziet, kan je zelf een aantal voorbeeldlijsten maken en a.d.h.v. de vooorbeelden nagaan wanneer het signaal een piek bereikt.
\end{oef}
\oplos{Er is een maximum voor element $u_i$ als en slechts als: $u_i > u_{i-1}$ en $u_{i} < u_{i+1}$.}

\begin{opdracht}{Automatische piekdetectie in App Inventor 2}
	\begin{minipage}{.1\linewidth}
		\includegraphics[width=1.5cm]{inputs/opdracht}
		\vspace{2.5cm}
	\end{minipage}
	\begin{minipage}{.8\linewidth}
		Schrijf een app die automatisch pieken detecteert in een lijst. 
		Hou bij \begin{enumerate}
			\item op welk volgnummer $i$ er zich een piek voordoet, en 
			\item wat het totaal aantal pieken is.
		\end{enumerate} 
		
		Gebruik labels om die resultaten op je scherm weer te geven ter controle.
	\end{minipage}
%	\begin{minipage}{.3\linewidth}
%		\gewonefiguur{width=5cm}{module2/intrestApp}
%	\end{minipage}
	
	\begin{tabular}{c|L{8cm}|L{8cm}}
		& \multicolumn{1}{>{\centering\arraybackslash}m{80mm}|}{\textbf{DENK}}  & \multicolumn{1}{>{\centering\arraybackslash}m{80mm}}{\textbf{DOE}}  \\
		\hline
		1 & & Cre\"eer een nieuw project, genaamd \textquotedblleft Automatische piekdetectie\textquotedblright.  \\
		\hline
		2 & \emph{Ontwerper} view: & \\
		&  1. Welke componenten heb je nodig? & \vspace{2cm} \\
		&  2. Welke naam geef je aan iedere component? & \vspace{2cm} \\
		\hline
		3 & \emph{Blokken} view: & \\
		&  1.  Gebruik eerst een lijst die je zelf ingeeft. Hoe ga je pieken detecteren?\vspace{.3cm} & \emph{Hint:} Gebruik \hl{\texttt{Lijsten $>$ maak een lijst}} om de lijst aan te maken. \\
		& 2. Doen de pieken zich voor op de plek die je verwacht? Pas je code aan tot je het verwachte resultaat bekomt. \vspace{.3cm} & \\
		& 3. Gebruik nu een lijst met reeds eerder berekende roodwaarden, die we in de vorige module hebben opgeslaan in een CSV-bestand. \newline 
		Indrukken van de knop \texttt{Lees CSV} resulteert in het inlezen van het CSV-bestand. Kijk indien nodig even terug in je vorige project waar en onder welke naam je het CSV-bestand hebt opgeslaan. \vspace{.3cm} & \emph{Hint: } Gebruik een \hl{\texttt{Bestand}}-component om het CSV-bestand in te lezen.\\ 
		& 4. Als het CSV-bestand is ingelezen, moet de inhoud omgezet worden in een lijst en moet het resultaat in een lijst geplaatst worden. Geef die lijst weer in een label en vergelijk de tekst met het CSV-bestand op een computer om te controleren dat dit correct gebeurde.\vspace{.3cm} & \\
		& 5. Gebruik de code uit stap 1 om automatisch pieken te detecteren in de lijst met roodwaarden. Komt dit overeen met het aantal pieken dat jij telt? Waarom (niet)? & \\
		
	\end{tabular}
	
	\begin{opmerking}
		Vaak is de lijst met roodwaarden een beetje \textquotedblleft ruizig\textquotedblright.
		%TODO voorbeeld van ruizige vs zuivere grafiek
		Dit zorgt ervoor dat de app meer pieken detecteert dan wij zouden doen. Soms zien we een klein piekje in het signaal, dat we negeren, omdat het maar een kleine piek is, of omdat de piek te snel na een grote hartslagpiek komt. 
		
		\begin{itemize}
			\item We moeten de app dus leren dat sommige pieken moeten genegeerd worden, wanneer de piek niet echt hoog is, of heel snel komt. Hoe zou jij dit implementeren? Je hoeft dit niet meteen te doen, maar zorg wel dat je een plan hebt.
			
			\item De makkelijkste manier om de kleine piekjes te negeren is door de lijst met roodwaarden een beetje \textquotedblleft op te kuisen\textquotedblright. De lijst met roodwaarden noemen we het \emph{signaal}. De aanwezige ruis is ongewenst. De ruis verandert vaak sneller dan het signaal zelf. Hier kunnen we handig gebruik van maken door die veranderingen op te sporen en weg te werken. Een eenvoudige manier om dit te doen bestaat erin opeenvolgende roodwaarden \textquotedblleft uit te middelen\textquotedblright. Hiermee bedoelen we dat we telkens een element uit de lijst te vervangen door het gemiddelde van een aantal elementen uit de lijst.
			%TODO figuur invoegen ter illustratie
			Dit proces heet \emph{laagdoorlaatfiltering}: we filteren de snelle veranderingen weg uit het signaal. Omdat de frequentie van de ruis hoger ligt dan de frequentie van het signaal, verwijzen we naar de ruis als \textquotedblleft hoge frequenties \textquotedblright. Door enkel de lage frequenties in beschouwing te nemen, negeren we de ruis en weten we dat we het signaal zelf aan het bestuderen zijn.
			Deze manier van werken heet \emph{laagdoorlaatfiltering}.
			
			Na het uitmiddelen van opeenvolgende waarden blijven mogelijk nog kleine piekjes over. Die overblijvende kleine piekjes kunnen we wegwerken door de getallen af te ronden. Op die manier blijven enkel de voldoende grote pieken over.
		\end{itemize}
	\end{opmerking}

	\begin{tabular}{c|L{8cm}|L{8cm}}
	& \multicolumn{1}{>{\centering\arraybackslash}m{80mm}|}{\textbf{DENK}}  & \multicolumn{1}{>{\centering\arraybackslash}m{80mm}}{\textbf{DOE}}  \\
	\hline
	& 5. Schrijf code om opeenvolgende roodwaarden uit te middelen en het resultaat af te ronden. 
	\begin{itemize}
		\item Maak een nieuwe lijst, bvb. \texttt{uitgemiddeldeRoodwaardenLijst}, waarin elk element een gemiddelde is van vier opeenvolgende elementen uit de originele lijst. 
		\item Herinner je dat we een gemiddelde berekenen als een som van elementen, gedeeld door het aantal elementen dat we optellen.
		Voor het gemiddelde van 4 willekeurige getallen:
		\begin{eqnarray*}
		\overline{a_4} &=& \text{Gemiddelde van $a_1$, $a_2$, $a_3$, $a_4$} \\
		&=& \frac{a_1+a_2+a_3+a_4}{4} = \frac{\sum_{i=1}^{4} a_i}{4}
		\end{eqnarray*}
	\end{itemize}
	&\emph{Hint:} gebruik een \hl{\texttt{Voor elke}}-lus om de elementen van de nieuwe lijst 	\texttt{uitgemiddeldeRoodwaardenLijst} te berekenen. \\
	& 6. Gebruik de code uit stap 1 om de pieken in de nieuwe lijst \texttt{uitgemiddeldeRoodwaardenLijst} te berekenen. \begin{itemize}
		\item Bekom je een aanvaardbaar resultaat? 
		\item Is het nodig meer of minder elementen van de originele lijst uit te middelen voor een aanvaardbaar resultaat?
	\end{itemize} & \\
	\end{tabular}

	\begin{opmerking}
	We spreken hier over een \emph{aanvaardbaar} resultaat en niet over een \emph{correct} resultaat. Het is goed mogelijk dat het aantal pieken dat je code detecteert licht afwijkt van het aantal pieken dat jij telt. Hoe sterk de afwijking is, hangt af van de kwaliteit van de opgenomen video. In Module 3 Sectie \ref{sec:Mod3_Sec3} gingen jullie al na welke factoren invloed hebben op de kwaliteit van de video.
	\end{opmerking}

	\begin{tabular}{c|L{8cm}|L{8cm}}
	& \multicolumn{1}{>{\centering\arraybackslash}m{80mm}|}{\textbf{DENK}}  & \multicolumn{1}{>{\centering\arraybackslash}m{80mm}}{\textbf{DOE}}  \\
	\hline
	& 7. Als je tevreden bent over de automatische piekdetectie, kan je het aantal pieken gedetecteerd tijdens de duur van de video herleiden naar een waarde voor de hartslag door na te gaan hoeveel pieken er zich per minuut voordoen. \newline
	\begin{itemize}
		\item Gebruik de frame rate en het aantal elementen in de lijst met roodwaarden om de duur van de video te berekenen. 
		\item Gebruik de regel van drie om het aantal pieken gedurende de videoduur te herleiden naar het aantal pieken per seconde. Dit is de gemeten hartslag. 
		\item Geef die waarde weer in een label op het scherm. 
		\item Bekom je een realistische waarde voor de gemeten hartslag?
	\end{itemize} & \\
	\end{tabular}

\opdrachteindbalk
\end{opdracht}

%%%%%%%%%%%%%%%%%%%%%%%%%%%%%%%%%%%%%%%%%%%%%%%%%%%%%%%%%%%%%%%%%%%%%%%
\section{Intermezzo: differentiequoti\"ent}
\label{sec:Mod4_Sec2}
%%%%%%%%%%%%%%%%%%%%%%%%%%%%%%%%%%%%%%%%%%%%%%%%%%%%%%%%%%%%%%%%%%%%%%%
%
\emph{Bedoeld voor leerlingen uit de 3e graad.}

Om te bepalen of het signaal een piek bereikt, maakte je gebruik van een wiskundige uitdrukking gerelateerd aan het differentiequoti\"ent. We gaan hier dieper op dit concept in.

Stel dat $\{ u_1, u_2, \ldots, u_n\}$ een signaal is, bv. een lijst met roodwaarden. 

\begin{opmerking}
	We noemen dit een \emph{discreet} signaal (in de figuur hieronder in rood aangeduid): de meetwaarden (bv roodwaarden) zijn slechts op bepaalde tijdstippen bepaald. 
	
	Naast \emph{discrete} signalen bestaan ook \emph{continue} signalen (in de figuur hieronder in zwart aangeduid), waarvoor op elk moment een meetwaarde beschikbaar is. Het signaal loopt dus - zoals de naam het zegt - continu door.
	
	\gewonefiguur{width=\linewidth}{module4/discreetvscontinu}
	
%	Vaak spreekt men ook over \emph{digitale} en \emph{analoge} signalen. Er zijn wel overeenkomsten tussen \emph{digitale} en \emph{discrete} signalen, respectievelijk \emph{analoge} en \emph{continue} signalen, maar in principe zijn het verschillende zaken.
%	
%	\begin{tabular}{ccc}
%	DISCREET & $\leftarrow \rightarrow$ & DIGITAAL \\
%	Overeenkomsten & & \\
%	Verschillen & & \\
%	CONTINU & $\leftarrow \rightarrow$ & ANALOOG \\
%	Overeenkomsten & & \\
%	Verschillen & & \\
%	\end{tabular}
\end{opmerking}

\begin{oef}
	In de lessen wiskunde leerde je al over functies. Een grafiek van de functie met functievoorschrift $y=f(x)$ toont hoe de (afhankelijke) grootheid $y$ verandert i.f.v. de (onafhankelijke) grootheid $x$. Vaak kan de grootheid $x$ een continu bereik van waarden aannemen. Spreken we in dat geval over een discreet of continu signaal?
\end{oef}
\oplos{Als de grootheid $x$ een continu bereik van waarden kan aannemen, spreken we over een continu signaal.}

De signalen waar we hier mee werken, kan je dus enigszins vergelijken met functies die je kent uit de wiskundelessen. 
Het verschil is dat we hier slechts op discrete momenten een functiewaarde kunnen bepalen, terwijl in de wiskundeles meestal met continue signalen gewerkt wordt.

We gaan terug naar ons discreet signaal $\{ u_1, u_2, \ldots, u_n\}$. De uitdrukking die je kan gebruiken om na te gaan of een piek bereikt was is:

\begin{equation*}
u_i-u_{i-1}
\end{equation*}

Immers, als $u_{i}-u_{i-1}>0$ en $u_{i+1}-u_{i}<0$ dan weten we dat er zich een lokaal maximum, of een piek, voordoet op het tijdstip $i$.

Deze uitdrukking is sterk gerelateerd aan het differentiequoti\"ent $D$ voor een functie $f$ in het punt $(a,f(a))$:

\begin{equation}
D = \frac{f(a+\Delta x) - f(a)}{\Delta x}
\end{equation}

Voor zeer kleine waarden van $\Delta x$ benadert het differentiequoti\"ent de richtingsco\"effici\"ent van de raaklijn in het punt $(a,f(a))$. 
We bepalen eerst twee punten $P$, met co\"ordinaten $(a,f(a))$, en $Q$, met co\"ordinaten $(a+\Delta x,f(a+\Delta x))$, op de grafiek van $f(x)$ en tekenen de koorde tussen beide punten.
De richtingsco\"effici\"ent van de koorde wordt gegeven door de tangens van de hoek die de koorde maakt met de horizontale. 
Die hoek noemen we $\alpha$.
De tangens van $\alpha$ is de verhouding van de lengte van de overstaande rechthoekszijde op de lengte van de aanliggende rechthoekszijde:
\begin{equation*}
\tan \alpha = \frac{f(a+\Delta x)-f(a)}{\Delta x}
\end{equation*}

\gewonefiguur{width=.7\linewidth}{module4/mooc_diffquot_raaklijn}

Hoe kleiner $\Delta x$, hoe dichter het punt $Q$ bij het punt $P$ ligt. In de limiet, wanneer $\Delta x$ nadert naar nul, benadert de richtingsco\"effici\"ent van de koorde de richtingsco\"efficient van de raaklijn in $P$ aan de grafiek van $f(x)$.

Hier \footnote{\url{http://tinyurl.com/y3zzjgv8}} vind je een mooie animatie terug die dit illustreert.

\gewonefiguur{width=4cm}{module4/qrcode_mooc_diffquot_raaklijn}

We zeggen ook weleens dat de afgeleide van de functie $f$ in het punt $(a,f(a))$ gelijk is aan 

\begin{eqnarray*}
f'(a) &=& \lim\limits_{\Delta x \to 0} D \\
&=& \lim\limits_{\Delta x \to 0} \frac{f(a+\Delta x)-f(a)}{\Delta x}
\end{eqnarray*}

Uit de wiskundelessen weet je dat een functie een extremum (= een maximum of een minimum) bereikt waar de afgeleide gelijk is aan nul. Om een \emph{minimum} te vinden zoeken we voor welke waarde(n) van $x$ de functie overgaat van dalend (afgeleide $f'(x)< 0$) naar stijgend (afgeleide $f'(x)> 0$). Om een \emph{maximum} te vinden zoeken we voor welke waarde van $x$ de functie overgaat van stijgend (afgeleide $f'(x)>0$) naar dalend (afgeleide $f'(x)<0$).

\gewonefiguur{width=\linewidth}{module4/1_9_fig2}

	\begin{center}
	\begin{tabular}{c|ccccc}
		$x$ & & 0 & & 2 & \\
		\hline 
		$f'(x)$ & + & 0 & - & 0 & + \\
		\hline
		$f(x)$ & $\nearrow$  & $\frac{0}{\text{max}}$ & $\searrow$ & $\frac{-4}{\text{min}}$ & $\nearrow$
	\end{tabular}
\end{center}


Wat we in dit project doen, is zeer gelijkaardig: we kijken voor welke waarde van $i$ de waarde van $u_{i}-u_{i-1}$ overgaat van een positieve naar een negatieve waarde om een piek (= een maximum) te detecteren. 

\begin{oef}
	Nu we achterhaald hebben hoe onze aanpak in dit project vergelijkbaar is met afgeleiden en extrema-vraagstukken, kunnen we ons ook afvragen welke verschillen bestaan. Som op welke verschillen je ziet.
\end{oef}
\oplos{\begin{itemize}
		\item Bij afgeleiden gebruiken we continue signalen, en hier discrete.
		\item Bij afgeleiden is er nog een deling door $\Delta h$. Dit laten we hier achterwege. 
		 \begin{opmerking}
		 	Merk op dat je ook de uitdrukking $u_{i}-u_{i-1}$ zou kunnen delen door de tijdsduur tussen 2 foto's. Die tijdsduur is $\Delta i=1/fs$, het omgekeerde van de frame rate. 
		 	
		 	Die deling heeft geen invloed op het resultaat (de gedetecteerde pieken), omdat we zoeken voor welke waarde van $i$ de uitdrukking $u_{i}-u_{i-1}$ of $\frac{u_{i}-u_{i-1}}{\Delta i} = [ u_{i}-u_{i-1} ] fs$  overgaat van een positieve naar een negatieve waarde. Die extra (positieve) factor $fs$ be\"invloedt het teken, en dus ook de uitkomst, niet.
		 \end{opmerking}
	\end{itemize}}

%%%%%%%%%%%%%%%%%%%%%%%%%%%%%%%%%%%%%%%%%%%%%%%%%%%%%%%%%%%%%%%%%%%%%%%
\section{Intermezzo: Frequentieanalyse}
\label{sec:Mod4_Sec3}
%%%%%%%%%%%%%%%%%%%%%%%%%%%%%%%%%%%%%%%%%%%%%%%%%%%%%%%%%%%%%%%%%%%%%%%
%
\emph{Bedoeld voor leerlingen uit de 3e graad.}
\footnote{Gebaseerd op de Junior College STEM over cochleaire implantaten.}

Je zal al ondervonden hebben dat de nauwkeurigheid van de hartslagmeting afhangt van de kwaliteit van het opgenomen filmpje. Als je vingertop of de smartphone bewoog tijdens de opname, is de grafiek van de roodwaarde in functie van de tijd ruiziger. Hoe ruiziger het signaal, hoe moeilijker de detectie van pieken wordt. En hoe moeilijker de piekdetectie, hoe meer kans op fouten en hoe onnauwkeuriger de hartslagmeting.

Er bestaat een andere manier om de hartslag te bepalen, die niet gebaseerd is op piekdetectie en daardoor iets nauwkeuriger is. Deze alternatieve manier is gebaseerd op frequentieanalyse.

\subsection{Geluid: zuivere en samengestelde tonen}

We nemen even afstand van onze hartslagmonitor. Om goed te kunnen uitleggen wat we met frequentieanalyse bedoelen, kijken we even naar geluid.

Geluid is een kleine verandering in luchtdruk, die zich door de lucht voortplant. Elk luchtdeeltje ondervindt een trilling en geeft die trilling door aan de luchtdeeltjes in zijn buurt. De trilling van een luchtdeeltje in functie van de tijd kan er uitzien zoals hieronder links.

\gewonefiguur{width=\linewidth}{module4/sinusTrilling}

De trilling hierboven (links) is een harmonische trilling, die als een sinusvormig signaal voorgesteld wordt en met als functievoorschrift 

\begin{equation*}
y(t) = A \sin(2\pi f t).
\end{equation*}

\begin{itemize}
	\item Links is de trilling voorgesteld in het tijdsdomein: de uitwijking in functie van de tijd is een sinusoidaal verloop. 
	\item Rechts is de trilling voorgesteld in het frequentiedomein. Deze voorstelling noemen we dan ook het \emph{frequentiespectrum}. De frequentie van een trilling is gedefineerd als het aantal trillingen per seconde. De eenheid is $1/s$ of Hertz ($Hz$). Hoe we frequentiespectrum  bepalen, komt later aan bod.
\end{itemize}

Deze trilling is een \emph{zuivere toon}: er is slechts 1 frequentie aanwezig. In het frequentiespectrum zien we maar 1 lijn. Het is het geluid dat je hoort als je een stemvork aanslaat.

\begin{oef}
\begin{itemize}
	\item Wat is de uitwijking, of amplitude $A$ van de trilling? 
	\item Hoe verandert het geluid als de amplitude verhoogt?
	\item Wat is de frequentie $f$ van de trilling? Vergeet de eenheid niet!
	\item Hoe verandert het geluid als de frequentie verhoogt?
	\item De periode is $T=1/f$ is het tijdsverschil tussen 2 opeenvolgende maximale (of minimale) uitwijkingen. Het is dus de tijd die 1 volledige trilling inneemt. Wat is de periode van de gegeven trilling?
\end{itemize}
\end{oef}
\oplos{\begin{itemize}
		\item De amplitude van de trilling is 2. Een eenheid werd niet gegeven.
		\item Het geluid wordt luider.
		\item De frequentie is 10 Hz. Dit kan je afleiden uit de tijdsgrafiek: we tellen 10 trillingen op $1~s$. Je kan dit ook aflezen op het frequentiespectrum: er is een grote piek op $10~Hz$.
		\item Het geluid klinkt hoger.
		\item De periode is $\frac{1}{10~Hz}=0.1~s$.
	\end{itemize}
}

Naast zuivere tonen bestaan ook samengestelde tonen. De meeste geluiden die je kent, zijn samengestelde tonen, waarin meerdere frequenties voorkomen. Hieronder zie je een voorbeeld.

\gewonefiguur{width=\linewidth}{module4/samengesteldeToon}

Het functievoorschrift van de afgebeelde trilling is 

\begin{equation*}
y(t) = 0.5 \sin(2 \pi 10 t) + 1 \sin(2 \pi 20 t) + 1 \sin(2 \pi 30 t).
\end{equation*}

De amplitude en frequentie van de samengestelde toon kan je uit het functievoorschrift afleiden: $A_1 = 0.5$, $f_1 = 10$ Hz, $A_2 = 1$, $f_2 = 20$ Hz, $A_3 = 1$ en $f_3 = 30$ Hz. De waarden van de aanwezige frequenties en amplitudes afleiding uit de linkergrafiek in het tijdsdomein is niet zo eenvoudig afgeleid worden. Het frequentiespectrum komt hiervoor veel beter van pas: de samengestelde trilling is opgebouwd uit drie zuivere trillingen, \'e\'en op 10 Hz met amplitude 0.5, \'e\'en op 20 Hz met amplitude 1 en \'e\'en op 30 Hz met amplitude 1.

In het algemeen kan een samengestelde toon geschreven worden als een som van $N$ zuivere tonen, elk met amplitude $A_n$ en frequentie $f_n$:

\begin{equation*}
y(t) = \sum_{n=1}^{N} A_n \sin(2 \pi f_n t).
\end{equation*}

\subsection{Frequentieanalyse: het idee van Fourier}

Meer algemeen kan je om het even welk periodiek signaal opbouwen door sinussen met veelvouden van een bepaalde grondfrequentie $f_0$ en verschillende amplituden op te tellen. Naarmate je meer termen optelt, krijgt het signaal een strakkere vorm.

Je kan elk periodiek signaal dus schrijven als een reeks

\begin{equation*}
y(t) = a_0 + \sum_{k=1}^{\infty} (a_k \cos(2 \pi k f_0 t) + b_k \sin(2 \pi k f_0 t)).
\end{equation*}

Frequentieanalyse of fourieranalyse is het ontleden van een oneindig lang periodiek signaal in de verschillende frequentiecomponenten waaruit het bestaat. 

Fourieranalyse bestaat er dus in de co\"effici\"enten $a_0$, $a_k$ en $b_k$ te zoeken, die bij elke frequentie $k f_0$ horen.

%TODO uitleg hoe dit te berekenen. Halen uit JC CI.

Fourieranalyse op geluidsignalen toont ons welke frequenties in een geluid voorkomen.

\subsection{Toepassing: fourieranalyse voor onze hartslagmonitor}

Terug naar onze hartslagmonitor. Wij zijn op zoek naar onze hartslag: het aantal keer dat ons hart klopt per minuuut. Dit is eigenlijk ook een soort van frequentie: het kloppen van ons hart kunnen we vergelijken met een geluidstrilling. Elke keer ons hart samentrekt, bevindt het bloed in onze vingertop zich iets verder van de huid en is de roodwaarde in onze video minimaal. En elke keer ons hart maximaal uitgezet is, bevindt het bloed in onze vingertop zich iets dichter bij huid en is de roodwaarde in onze video maximaal. Onze hartslag is dus de frequentie waarmee ons hart klopt en is meteen ook de frequentie waarmee de roodwaarde in de opgenomen video fluctueert. Als we dus het frequentiespectrum van ons roodwaarde-signaal kunnen opstellen, kunnen we uit de pieken hierin de hartslag afleiden.

\begin{opdracht}{Fourieranalyse in App Inventor 2}
	\begin{minipage}{.1\linewidth}
		\includegraphics[width=1.5cm]{inputs/opdracht}
		\vspace{0cm}
	\end{minipage}
	\begin{minipage}{.5\linewidth}
		Gebruik Fourieranalyse om de hartslag in een roodwaarde signaal te bepalen.
	\end{minipage}
%	\begin{minipage}{.3\linewidth}
%		\gewonefiguur{width=5cm}{module2/intrestApp}
%	\end{minipage}
	
	\begin{tabular}{c|L{8cm}|L{8cm}}
		& \multicolumn{1}{>{\centering\arraybackslash}m{80mm}|}{\textbf{DENK}}  & \multicolumn{1}{>{\centering\arraybackslash}m{80mm}}{\textbf{DOE}}  \\
		\hline
		1 & & Cre\"eer een nieuw project, genaamd \textquotedblleft IL\_hartslagmonitor\_fourieranalyse\textquotedblright.  \\
		\hline
		2 & \emph{Ontwerper} view: & \\
		& 1. Voorzie een aantal knoppen: \texttt{leesCSV}, \texttt{berekenDFT} en \texttt{grafiekDFT}. \vspace{.3cm}& \\
		& 2. Voorzie ook een textbox waarin de gebruiker de frame rate ($fs$) kan aanpassen en een knop \texttt{fsVeranderd} waarmee de gebruiker kan aangeven dat hij de frame rate aangepast heeft. \vspace{.3cm} & \\
		3 & \emph{Blokken} view: & \\
		& 1. Als de knop \texttt{fsVeranderd} aangeklikt wordt, moet een globale variabele aangepast worden aan de inhoud van de textbox. \vspace{.3cm}& \\
		& 2. Gebruik nu een lijst met berekende roodwaarden, die we in de vorige module hebben opgeslaan in een CSV-bestand.  Wanneer een knop \texttt{leesCSV} ingedrukt wordt, moet het CSV-bestand dat je opgeslaan hebt in de vorige module ingelezen worden. & Gebruik zoveel mogelijk code uit vorige opdrachten. Code kopi\"eren tussen verschillende projecten doe je door de blokken code binnen het ene project vanuit het witte venster naar de rugzak rechtsboven te slepen en die vervolgens binnen een ander project vanuit de rugzak naar het witte venster te slepen. \newline
		Gebruik een \hl{\texttt{Bestand}}-component om het CSV-bestand in te lezen. \vspace{.3cm}\\
		& 3. Als het CSV-bestand is ingelezen, moet de inhoud omgezet worden in een lijst en in een lijst geplaatst worden. Geef die lijst weer in een label en vergelijk de tekst met het CSV-bestand op een computer om te controleren dat dit correct gebeurde. \vspace{.3cm} & \\
%		& & 4. Kopieer de code uit een van de vorige opdrachten om opeenvolgende roodwaarden uit te middelen en af te ronden. Maak een nieuwe lijst, bvb. \texttt{uitgemiddeldeRoodwaardenLijst}, waarin elk element een afgerond gemiddelde is van vier opeenvolgende elementen uit de originele lijst. \\
	\end{tabular}

	\begin{tabular}{c|L{8cm}|L{8cm}}
	& \multicolumn{1}{>{\centering\arraybackslash}m{80mm}|}{\textbf{DENK}}  & \multicolumn{1}{>{\centering\arraybackslash}m{80mm}}{\textbf{DOE}}  \\
	\hline
		& & 5. De implementatie van de Fourieranalyse hebben we al voor jullie gedaan. Voeg daarom de extensie \hl{\texttt{FourierTransformatie}} toe aan je app. \vspace{.3cm}\\
		& 6. Voer een Fourieranalyse uit te voeren wanneer de knop \emph{berekenFourierTF} aangeklikt wordt. Het resultaat is een lijst \texttt{fourierTFLijst}. \newline
		Wat gebruik je als input voor de Fourieranalyse? \vspace{.3cm}& Gebruik het blok \hl{\texttt{FourierTransformatie $>$ FourierTFBerekenen}} \\
		& 7. Verwijder het eerste element uit de lijst. Dit is nodig om een correct frequentiespectrum te kunnen maken. \vspace{.3cm}& \\
		& 8. Indrukken van de knop \texttt{grafiekDFT} dient het frequentiespectrum te tonen.\vspace{.3cm} & Gebruik hiervoor het blok \hl{\texttt{GrafiekMaken $>$ MaakGrafiek}} en een \hl{\texttt{Webviewer}}. \\
		& 9. De titel van de grafiek en van de assen en het label kies je zelf. \vspace{.3cm}& 9. Maak de grafiek eerst met \texttt{stapgrootteX} gelijk aan 1. \\
		& 10. Op de $x$-as staan de mogelijke hartslagfrequenties. De sprong tussen twee opeenvolgende hartslagfrequenties is $\frac{60 \cdot fs}{\text{aantal roodwaarden}}$. \vspace{.3cm} & 10. Pas de \texttt{stapgrootteX} hieraan aan. \\
		& 11. Welke hartslag meten we nu op? \newline \begin{itemize}
			\item De grootste piek in de grafiek toont welke frequentie \textquotedblleft het meeste in het signaal aanwezig is\textquotedblright. De $x$-waarde van de grootste piek toont dus de meest waarschijnlijke hartslag.
			\item Komt de hoogste piek overeen met de hartslag die je zelf berekende? En met de hartslag die je app berekende via automatische piekdetectie?
		\end{itemize}
		& 11. Je kan de $x$- en $y$- waarde van een punt op de grafiek zien door erop te klikken.  \\
	\end{tabular}

	\begin{opmerking}
	Het is mogelijk dat de grootste piek zichtbaar is voor een heel kleine $x$-waarde. In dat geval is de piek wellicht veroorzaakt door ruis in het signaal. Komt de 2e grootste piek wel in de buurt bij een meer waarschijnlijke hartslag?
	\end{opmerking}

	Voor de roodwaarden in het voorbeeld CSV-bestand zou je een grafiek moeten zien zoals hieronder: 
	
	\gewonefiguur{width=\linewidth}{module4/redValuesFFT}
	
	In dit geval is de meest waarschijnlijke hartslag ongeveer 75 bpm.
	

	\begin{itemize}
		\item Wat is de stapgrootte waarmee de frequenties op de $x$-as verspringen? 
		\item Wat zegt dit over de nauwkeurigheid van de hartslagmeting?
		\item Heeft de stapgrootte verband met een variabele die we eerder al gebruikten? Welke? 
		\item Hoe kan je de nauwkeurigheid van de hartslagmeting verbeteren?
	\end{itemize}

	Test de werking van je app grondig. 

\opdrachteindbalk

\end{opdracht}






