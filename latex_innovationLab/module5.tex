%%%%%%%%%%%%%%%%%%%%%%%%%%%%%%%%%%%%%%%%%%%%%%%%%%%%%%%%%%%%%%%%%%%%%%%
%%%%%%%%%%%%%%%%%%%%%%%%%%%%%%%%%%%%%%%%%%%%%%%%%%%%%%%%%%%%%%%%%%%%%%%
\deelmetoef{Module 5}{Bouwen van de complete app}{Module 5. Bouwen van de complete app}{Oplossingen module 5}{Oplossingen module 5}
%%%%%%%%%%%%%%%%%%%%%%%%%%%%%%%%%%%%%%%%%%%%%%%%%%%%%%%%%%%%%%%%%%%%%%%
%%%%%%%%%%%%%%%%%%%%%%%%%%%%%%%%%%%%%%%%%%%%%%%%%%%%%%%%%%%%%%%%%%%%%%%

\begin{samenvatting}
	Op dit moment heb je al alle stappen overlopen die nodig zijn om je hartslag te meten m.b.v. je smartphone: (1) een video opnemen, (2) de video splitsen in afbeeldingen, (3) de gemiddelde roodwaarde voor elke afbeelding berekenen, (4) een graiek van de gemiddelde roodwaarde i.f.v. de tijd maken, (5) automatische piekdetectie en berekening van de hartslag, en eventueeel (6) frequentie-analyse m.b.v. de Fouriertransformatie. 
	Elk bouwblok uit het schema uit Module 1 hebben we apart ge\"implementeerd, waarbij we abstractie maakten van wat in eerdere en latere bouwblokken moest gebeuren.
	Bij de implementatie van de functionaliteit van een bouwblok houden we enkel rekening met welke inputs en outputs dat bouwblok heeft; verder negeren we de complexiteit en de functionaliteit van de andere bouwblokken.
	Die segmentering (het opdelen van een probleem in segmenten of deelproblemen) laat ons toe onze volledige aandacht te richten op 1 deelprobleem van het meer complexe hartslagmonitor-probleem.
	Natuurlijk moet alles op een bepaald moment samenkomen, en dat moment is nu gekomen.
\end{samenvatting}
%

%%%%%%%%%%%%%%%%%%%%%%%%%%%%%%%%%%%%%%%%%%%%%%%%%%%%%%%%%%%%%%%%%%%%%%%
\section{Video opnemen}
\label{sec:Mod5_Sec1}
%%%%%%%%%%%%%%%%%%%%%%%%%%%%%%%%%%%%%%%%%%%%%%%%%%%%%%%%%%%%%%%%%%%%%%%
%

%%%%%%%%%%%%%%%%%%%%%%%%%%%%%%%%%%%%%%%%%%%%%%%%%%%%%%%%%%%%%%%%%%%%%%%
\section{Video naar afbeeldingen m.b.v. app VideoNaarAfbeeldingen}
\label{sec:Mod5_Sec2}
%%%%%%%%%%%%%%%%%%%%%%%%%%%%%%%%%%%%%%%%%%%%%%%%%%%%%%%%%%%%%%%%%%%%%%%
%

%%%%%%%%%%%%%%%%%%%%%%%%%%%%%%%%%%%%%%%%%%%%%%%%%%%%%%%%%%%%%%%%%%%%%%%
\section{Gemiddelde roodwaarde voor elke afbeelding berekenen}
\label{sec:Mod5_Sec3}
%%%%%%%%%%%%%%%%%%%%%%%%%%%%%%%%%%%%%%%%%%%%%%%%%%%%%%%%%%%%%%%%%%%%%%%
%

%%%%%%%%%%%%%%%%%%%%%%%%%%%%%%%%%%%%%%%%%%%%%%%%%%%%%%%%%%%%%%%%%%%%%%%
\section{Grafiek van de gemiddelde roodwaarde i.f.v. de tijd}
\label{sec:Mod5_Sec4}
%%%%%%%%%%%%%%%%%%%%%%%%%%%%%%%%%%%%%%%%%%%%%%%%%%%%%%%%%%%%%%%%%%%%%%%
%

%%%%%%%%%%%%%%%%%%%%%%%%%%%%%%%%%%%%%%%%%%%%%%%%%%%%%%%%%%%%%%%%%%%%%%%
\section{Automatische piekdetectie en berekening van de hartslag}
\label{sec:Mod5_Sec5}
%%%%%%%%%%%%%%%%%%%%%%%%%%%%%%%%%%%%%%%%%%%%%%%%%%%%%%%%%%%%%%%%%%%%%%%
%

%%%%%%%%%%%%%%%%%%%%%%%%%%%%%%%%%%%%%%%%%%%%%%%%%%%%%%%%%%%%%%%%%%%%%%%
\section{Extra: frequentie-analyse m.b.v. Fouriertransformatie}
\label{sec:Mod5_Sec6}
%%%%%%%%%%%%%%%%%%%%%%%%%%%%%%%%%%%%%%%%%%%%%%%%%%%%%%%%%%%%%%%%%%%%%%%
%

