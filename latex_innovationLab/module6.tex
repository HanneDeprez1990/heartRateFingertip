%%%%%%%%%%%%%%%%%%%%%%%%%%%%%%%%%%%%%%%%%%%%%%%%%%%%%%%%%%%%%%%%%%%%%%%
%%%%%%%%%%%%%%%%%%%%%%%%%%%%%%%%%%%%%%%%%%%%%%%%%%%%%%%%%%%%%%%%%%%%%%%
\deelmetoef{Module 6}{Experimenteren met de app}{Module 6. Experimenteren met de app}{Oplossingen module 6}{Oplossingen module 6}
%%%%%%%%%%%%%%%%%%%%%%%%%%%%%%%%%%%%%%%%%%%%%%%%%%%%%%%%%%%%%%%%%%%%%%%
%%%%%%%%%%%%%%%%%%%%%%%%%%%%%%%%%%%%%%%%%%%%%%%%%%%%%%%%%%%%%%%%%%%%%%%

\begin{samenvatting}
	Nu we een app hebben gemaakt waarmee we onze hartslag kunnen meten is het tijd om de app te evalueren. Net zoals bij elk ontwikkeld product zijn er wellicht verbeteringen mogelijk. In deze module gaan we met de app aan de slag en vergelijken we de app met alternatieven om je hartslag te meten, met als bedoeling eventuele fouten te achterhalen en verbeterpunten te ontdekken. Kritisch stilstaan bij je ontwerp en bij je product is noodzakelijk als je je product wilt blijven verbeteren.
\end{samenvatting}
%

%%%%%%%%%%%%%%%%%%%%%%%%%%%%%%%%%%%%%%%%%%%%%%%%%%%%%%%%%%%%%%%%%%%%%%%
\section{Invloed van sport en rust}
\label{sec:Mod6_Sec2}
%%%%%%%%%%%%%%%%%%%%%%%%%%%%%%%%%%%%%%%%%%%%%%%%%%%%%%%%%%%%%%%%%%%%%%%
%

Het is natuurlijk interessant om te weten wat je hartslag bij rust is, zoals besproken in Module \ref{sec:inl}, maar de meeste mensen gebruiken hartslagmeters om na te gaan wat je hartslag is bij inspanning.

\begin{opdracht}{Hartslag bij rust en na inspanning meten}
Meet je hartslag vlak voor en vlak na een inspanning. Zit je hartslag na inspanning binnen het aanbeloven hartslagbereik voor jouw leeftijd?

Verschilt je hartslag bij rust als je rechtop staat, rechtop zit of ligt? Waaraan zou dit kunnen liggen?
\end{opdracht}

%%%%%%%%%%%%%%%%%%%%%%%%%%%%%%%%%%%%%%%%%%%%%%%%%%%%%%%%%%%%%%%%%%%%%%%
\section{Vergelijking met bloeddrukmeter/hartslagmeter/smartwatch}
\label{sec:Mod6_Sec1}
%%%%%%%%%%%%%%%%%%%%%%%%%%%%%%%%%%%%%%%%%%%%%%%%%%%%%%%%%%%%%%%%%%%%%%%
%
Zoals we in Module \ref{sec:inl} besproken hebben, bestaan er alternatieven om je hartslag te meten. We spraken over bloeddrukmeters en hartslagmeters. Maar ook apps zoals de onze zijn al te verkrijgen in de Google Play Store.

\begin{opdracht}{Vergelijking met alternatieve hartslagmonitors}
Verzamel een aantal alternatieven voor onze hartslagmonitor-app. 

Som eerst op welke factoren je belangrijk vindt bij het gebruik van een hartslagmonitor. 

Ga na hoe goed de verschillende alternatieven naar jouw mening scoren voor elk van die factoren. Dit is een kwalitatieve evaluatie. Jij beoordeelt de kwaliteit van elk alternatief, op basis van jouw aanvoelen.

Misschien kan je sommige van die scores ook objectiever maken. Objectieve maatstaven zijn het tegengestelde van subjectieve maatstaven. Objectiever wil zeggen dat het oordeel minder afhankelijk is van jouw aanvoelen of van jouw persoonlijke mening, maar meer neutraal, zonder vooroordeel, kan vastgesteld worden. Vaak worden hiervoor kwantatieve maten gebruikt: dit zijn factoren die meetbaar zijn. 

Wellicht is de snelheid van de hartslagmeting voor jou belangrijk. Je subjectieve aanvoelen is wellicht dat de app die wij ontworpen hebben minder snel is dan een commercieel beschikbare hartslagmeter. (Dit is natuurlijk omdat het ook doenbaar moet zijn om de hartslagmonitor binnen een beperkt aantal lessen te maken, op basis van de kennis die jullie nu al hebben.) Maar als objectieve maat volstaat dit aanvoelen niet. Je zou met een chronometer kunnen nagaan hoelang de hartslagmeting duurt voor elk van de alternatieven. Dit is een objectieve maatstaf voor de snelheid van de hartslagmeting. 

Kan je ook voor de andere factoren die jij belangrijk vindt bepalen hoe je elke factor een objectieve score zou kunnen geven?
\end{opdracht}



%%%%%%%%%%%%%%%%%%%%%%%%%%%%%%%%%%%%%%%%%%%%%%%%%%%%%%%%%%%%%%%%%%%%%%%
\section{Evaluatie: wat kan beter?}
\label{sec:Mod6_Sec3}
%%%%%%%%%%%%%%%%%%%%%%%%%%%%%%%%%%%%%%%%%%%%%%%%%%%%%%%%%%%%%%%%%%%%%%%
%
\begin{opdracht}{Plus- en minputen en suggesties voor verbetering bepalen}
	Uit de vergelijking met de alternatieven heb je zelf kunnen ervaren hoe goed of slecht onze hartslagmonitor werkt. 

\begin{steroef}
	Op welke vlakken scoort de hartslagmonitor goed?
Wat kan beter? 
\end{steroef}

\begin{steroef}
	Als ontwerper moet je niet alleen bepalen wat de minpunten van je ontwerp zijn; je moet ook nadenken over hoe je het ontwerp kan verbeteren. Welke aanpassingen zou jij doen?
\end{steroef}

\begin{steroef}
	In een ideale wereld kan je uiteraard veel aanpassingen doen, maar als echte ontwerper ben je meestal meer beperkt. Welke beperkingen zie jij? 
\end{steroef}
\oplos{\begin{itemize}
		\item Budget
		\item Tijd
		\item Wiskundige achtergrond
		\item Beperkingen hardware: snelle vs trage smartphone, geheugen, wel of geen flash, \ldots
	\end{itemize}}

\begin{steroef}
	De beperkingen bepalen ook de haalbaarheid van de verbetering. Welke aanpassingen zijn haalbaar, welke zijn moeilijker te implementeren?
\end{steroef}

\begin{opmerking}
	Merk op dat voor dit probleem (net als voor veel andere problemen!) geen modeloplossing beschikbaar is. Vaak is een probleem \emph{open-ended}; dit wil zeggen dat niet op voorhand vastligt wat de beste oplossing voor het probleem is. Soms kan je door ervaring inschatten welke ontwerpen de beste zullen zijn, maar soms moet je het ook gewoon uitproberen en via evaluatie achterhalen wat goed en wat minder goed loopt en welke aanpassingen de kwaliteit van je producht ten goede be\"invloeden. Deze probleemoplossende vaardigheden zijn cruciaal in veel beroepen en vergen een zekere kritische houding en creativiteit.
\end{opmerking}
\end{opdracht}
